\documentclass{article}
\usepackage[a5paper]{geometry}
\usepackage{fontspec}
\usepackage[math]{blindtext}
\usepackage{unicode-math}
%	\usepackage{realnew}
\usepackage{luacode}
\begin{luacode*}
np = require "nodeprocess"
pagebuilder = require "pagebuilder"
parbuilder = require "parbuilder" 
-- luatexbase.add_to_callback("pre_linebreak_filter", 
-- function(head) 
--   pagebuilder.add(np:process_hlist(head))
--   return head
-- end, "Save paragraphs")
\end{luacode*}
\newtoks\oldoutput
\oldoutput=\expandafter{\the\output}
\output{%
\typeout{Output routine}
\luaexec{pagebuilder.add("\thepage",np:make_paragraphs(tex.box[255].list))}
\typeout{End output routine}
\the\oldoutput
}


\AfterEndDocument{%
	\luaexec{pagebuilder.save(tex.jobname..".pages")}
	\luaexec{parbuilder.save(tex.jobname..".pars")}
}
\begin{document}
Příliš žluťoučký kůň úpěl ďábelské ódy\footnote{pokus}. A $c=\sqrt{a^2+b^2}$. Trochu ten text prodloužím, aby mi taky někde vyšel konec stránky uvnitř odstavce. Očividně musím trochu víc, jeden řádek nestačil. No ale to musím trošku přidat, zatím to o další řádek text neprotáhlo. Je už to teď lepší? 

\section{pokus}

a další text

\[\frac{a+b}{c-d}\]

a ještě další
\blinddocument
\end{document}


